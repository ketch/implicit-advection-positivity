\documentclass[a4paper]{article}

%% Language and font encodings
\usepackage[english]{babel}
\usepackage[utf8x]{inputenc}
\usepackage[T1]{fontenc}

%% Sets page size and margins
\usepackage[a4paper,top=3cm,bottom=2cm,left=3cm,right=3cm,marginparwidth=1.75cm]{geometry}

%% Useful packages
\usepackage{amsmath,amsthm}
\usepackage{graphicx}
\usepackage[colorinlistoftodos]{todonotes}
\usepackage[colorlinks=true, allcolors=blue]{hyperref}
\newtheorem{theorem}{Theorem}

\newcommand{\dt}{\Delta t}

\title{Positivity of implicit discretizations of the advection equation}
\author{Yiannis Hadjimichael \and David I. Ketcheson \and Lajos Loczi}

\begin{document}
\maketitle

\section{Background and motivation}
Here we investigate the positivity of some discretizations of the advection equation
\begin{align} \label{advection}
U_t = a U_x.
\end{align}
The advection equation is a fundamental PDE, and herein we focus on
some of the simplest and most fundamental discretizations of it.

Any linear one-step discretization of a time-dependent PDE in one spatial
dimention with $m$ grid points in space yields an approximate solution given by
an iteration of the form
\begin{align} \label{M}
    U^{n+1} = Mu^n
\end{align}
where $M$ is a fixed $m\times m$ matrix.


\subsection{Discrete Fourier analysis}
We can (in principle) analyze positivity of any other combination of spatial and (one-step) time discretization in the same manner.  We
will have some circulant matrix $L = (1/m) F^* D F$ but where
the eigenvalues (entries of $D$) are different from the values
$\lambda_j = i\sin \theta_j$ above.  For instance, in a spectral collocation
method the eigenvalues will be simply $\lambda_j = i \theta_j$.  The time
discretization will possess a stability function $R(z)$, and
we are interested in positivity of the entries of the (real, circulant)
matrix $M=F^* R(\nu D) F$.  The first row of this matrix is given by

$$M_{1,j} = \frac{1}{m} \sum_{k=1}^m R(\nu\lambda_k) \exp(i(j-1)\theta_k).$$

Looking at it from the opposite direction, one could try to choose a spatial
discretization (by choosing the $\lambda_j$) and time discretization
(by choosing $R(z)$) so as to obtain a positive scheme, under some constraints on the accuracy.  To have order $p$ accuracy in time,
we require $R(z) = \exp(z) + O(z^{p+1})$, while to have order $q$
accuracy in space we require $\lambda_j = i \theta_j + O(m^{-(q+1)}).$


\section{Backward in time, centered in space}
A finite difference spatial discretization of \eqref{advection} yields a system
of ODEs
\begin{align} \label{semi-discrete}
    U'(t) = LU(t)
\end{align}
where $U$ is the vector whose entries are the values of the solution
at the grid points, as a function of time, and $U_x \approx \frac{1}{\Delta x} L u$.
We first consider the 3-point centered difference in space so that $L$ is a
circulant matrix with entries $(-1/2, 0, 1/2)$ on the central three diagonals.
It is straightforward to show that the system \eqref{semi-discrete} can
give negative values even if the initial values are positive.

If we use the backward Euler method in time, we have
$$u^{n+1} = u^n + \frac{\Delta t}{\Delta x} L u^{n+1}$$
or
$$u^{n+1} = (I-\nu L)^{-1} u^n,$$
where $\nu = \Delta t/\Delta x$ is the CFL number.  The method will be positivity preserving if the matrix
$(I-\nu L)^{-1}$ has non-negative entries.


The results we already have for backward Euler + centered differences go here.

\begin{theorem}
Consider the backward in time, centered in space discretization of the
advection equation with periodic boundary conditions and $m$ spatial grid
points.  The discretization takes the form \eqref{M},
where (i) if $m$ is even, then $M$ has at least one negative entry;
(ii) if $m$ is odd, then for $\dt>\dt_*$ all entries of $M$ are nonnegative.

(add details about $\dt_*$)
\end{theorem}

\subsection{Discrete Fourier analysis}
David's work here.

\subsection{Detailed analysis of the matrix entries}
Lajos' work here.

\section{Other discretizations}

\end{document}
